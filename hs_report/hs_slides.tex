\documentclass[
 size=14pt,
 paper=smartboard,  %a4paper, smartboard, screen
 mode=present, 		%present, handout, print
 display=slides, 	% slidesnotes, notes, slides
 style=tuliplab,  	% TULIP Lab style
 pauseslide,
 fleqn,leqno]{powerdot}


\usepackage{cancel}
\usepackage{caption}
\usepackage{stackengine}
\usepackage{smartdiagram}
\usepackage{attrib}
\usepackage{amssymb}
\usepackage{amsmath} 
\usepackage{amsthm} 
\usepackage{mathtools}
\usepackage{rotating}
\usepackage{graphicx}
\usepackage{boxedminipage}
\usepackage{rotate}
\usepackage{calc}
\usepackage[absolute]{textpos}
\usepackage{psfrag,overpic}
\usepackage{fouriernc}
\usepackage{pstricks,pst-3d,pst-grad,pstricks-add,pst-text,pst-node,pst-tree}
\usepackage{moreverb,epsfig,subfigure}
\usepackage{color}
\usepackage{booktabs}
\usepackage{etex}
\usepackage{breqn}
\usepackage{multirow}
\usepackage{natbib}
\usepackage{bibentry}
\usepackage{gitinfo2}
\usepackage{siunitx}
\usepackage{nicefrac}
%\usepackage{geometry}
%\geometry{verbose,letterpaper}
\usepackage{media9}
\usepackage{animate}
%\usepackage{movie15}
\usepackage{auto-pst-pdf}

\usepackage{breakurl}
\usepackage{fontawesome}
\usepackage{xcolor}
\usepackage{multicol}



\usepackage{verbatim}
\usepackage[utf8]{inputenc}
%\usepackage{dtk-logos}
\usepackage{tikz}
\usepackage{adigraph}
\usepackage{tkz-graph}
\usepackage{hyperref}
\usepackage{ulem}
\usepackage{pgfplots}
\usepackage{verbatim}
\usepackage{fontawesome}


\usepackage{todonotes}
% \usepackage{pst-rel-points}
\usepackage{animate}
\usepackage{fontawesome}

\usepackage{listings}
\lstset{frameround=fttt,
frame=trBL,
stringstyle=\ttfamily,
backgroundcolor=\color{yellow!20},
basicstyle=\footnotesize\ttfamily}
\lstnewenvironment{code}{
\lstset{frame=single,escapeinside=`',
backgroundcolor=\color{yellow!20},
basicstyle=\footnotesize\ttfamily}
}{}


\usepackage{hyperref}
\hypersetup{ % TODO: PDF meta Data
  pdftitle={Presentation Title},
  pdfauthor={Gang Li},
  pdfpagemode={FullScreen},
  pdfborder={0 0 0}
}


% \usepackage{auto-pst-pdf}
% package to show source code

\definecolor{LightGray}{rgb}{0.9,0.9,0.9}
\newlength{\pixel}\setlength\pixel{0.000714285714\slidewidth}
\setlength{\TPHorizModule}{\slidewidth}
\setlength{\TPVertModule}{\slideheight}
\newcommand\highlight[1]{\fbox{#1}}
\newcommand\icite[1]{{\footnotesize [#1]}}

\newcommand\twotonebox[2]{\fcolorbox{pdcolor2}{pdcolor2}
{#1\vphantom{#2}}\fcolorbox{pdcolor2}{white}{#2\vphantom{#1}}}
\newcommand\twotoneboxo[2]{\fcolorbox{pdcolor2}{pdcolor2}
{#1}\fcolorbox{pdcolor2}{white}{#2}}
\newcommand\vpspace[1]{\vphantom{\vspace{#1}}}
\newcommand\hpspace[1]{\hphantom{\hspace{#1}}}
\newcommand\COMMENT[1]{}

\newcommand\placepos[3]{\hbox to\z@{\kern#1
        \raisebox{-#2}[\z@][\z@]{#3}\hss}\ignorespaces}

\renewcommand{\baselinestretch}{1.2}


\newcommand{\draftnote}[3]{
	\todo[author=#2,color=#1!30,size=\footnotesize]{\textsf{#3}}	}
% TODO: add yourself here:
%
\newcommand{\gangli}[1]{\draftnote{blue}{GLi:}{#1}}
\newcommand{\shaoni}[1]{\draftnote{green}{sn:}{#1}}
\newcommand{\gliMarker}
	{\todo[author=GLi,size=\tiny,inline,color=blue!40]
	{Gang Li has worked up to here.}}
\newcommand{\snMarker}
	{\todo[author=Sn,size=\tiny,inline,color=green!40]
	{Shaoni has worked up to here.}}

%%%%%%%%%%%%%%%%%%%%%%%%%%%%%%%%%%%%%%%%%%%%%%%%%%%%%%%%%%%%%%%%%%%%%%%%
% title
% TODO: Customize to your Own Title, Name, Address
%
\title{Flip00 Group Weekly Report}
\author{
Han Sen
\\
\\Beijing Technology And Business University
\\
\\
}
\date{\gitCommitterDate}


% Customize the setting of slides
\pdsetup{
% TODO: Customize the left footer, and right footer
rf=\href{http://www.tulip.org.au}{
Last Changed by: \textsc{\gitCommitterName}\ (\gitAuthorDate)
},
cf={Flip00 Group Weekly Report},
}
%' \gitVtagn-\gitAbbrevHash\ '

\begin{document}

\maketitle

%%==========================================================================================
%% 目录页
\begin{slide}[toc=,bm=]{Overview}
\tableofcontents[content=currentsection,type=1]
\end{slide}
%%
%%==========================================================================================


%%==========================================================================================
%%
% 章节,标题页
\section{Weekly Report}
%%
%%==========================================================================================


%%==========================================================================================
%%
\begin{slide}{Learning Progress}

\bigskip

\twocolumn[
\savevalue{lfrheight}=5.2cm,
\savevalue{lfrprop}={
linestyle=solid,framearc=.2,linewidth=1pt},
rfrheight=\usevalue{lfrheight},
rfrprop=\usevalue{lfrprop}
]{
About Latex
\begin{itemize}
\item
\smallskip
Install and debug latexdiff
\smallskip
\item
\smallskip
Push to Github repository and get Gitinfo
\end{itemize}
}{
About Git
\begin{itemize}
\item
\smallskip
Install and debug Smartgit
\smallskip
\item
\smallskip
Get github support to unlock my account
\item
\smallskip
Set up a remote repository and push local changes
\end{itemize}
}


\end{slide}
%%
%%==========================================================================================


%%==========================================================================================
%%
\begin{slide}{Problems and difficulties}
\twotonebox {\rotatebox{90}{Problems}}{\parbox{.88\textwidth}
{
{About gitinfo2
\begin{itemize}
\item
When I use /gitvtagn, I get "None".
\end{itemize}
}
}}

\end{slide}
%%
%%==========================================================================================


\section{Next week's plan}


%%==========================================================================================
%%
\begin{slide}{Kaggle Competition Topics}
\begin{itemize}
\item
\href{https://www.kaggle.com/competitions/playground-series-s3e5/data}{
Ordinal Regression with a Tabular Wine Quality Dataset
}
\begin{itemize}
\item
There is a dataset describing the amount of various chemicals in wine and their effect on its quality. 
\item
Predict the quality of the wine using the given data.
\end{itemize}
\end{itemize}
\end{slide}
%%
%%==========================================================================================

\end{document}

\endinput
